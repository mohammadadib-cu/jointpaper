\pagenumbering{arabic}
\verso{First Author et al.}

%% Paper title
\newcommand{\papertitle}{Yield, morphology and maturity of carbon black made by methane pyrolysis in a shock tube}

\begin{frontmatter}
\title{
    \papertitle
}

\author[1]{First \snm{Author}\corref{cor1}}
\cortext[cor1]{Corresponding author: Address}
\emailauthor{email@email.com}{First Author}

\author[2]{Second \snm{Author}}

\author[3]{Third \snm{Author}}


\address{\textsuperscript{1} Address}


\begin{abstract}
Time-resolved measurements of gaseous species and carbon black (CB) were conducted during the pyrolysis of 5\% $\mathrm{CH_4}$ in Ar at post-reflected-shock pressures near $5$~atm over a temperature range of $1800$–$2400$~K, to better understand the carbon conversion flux from $\mathrm{CH_4}$ to gaseous intermediates and solid carbonaceous particles. Soot morphology was characterized using Transmission Electron Microscopy (TEM) across the studied temperature range. An imposed-pressure reactor model, coupled with detailed chemistry and a sectional population balance model (SPBM), was employed to simulate CB formation and growth. Simulations using the Caltech mechanism accurately predicted the $\mathrm{CH_4}$ conversion rate but underpredicted the carbon mass in acetylene ($\mathrm{C_2H_2}$) and overpredicted that in ethylene ($\mathrm{C_2H_4}$). The inception flux was identified as the primary factor influencing model predictions of soot volume fraction ($f_v$) and mean primary particle diameter ($d_p$). The model accounts for CB carbonization, which alters particle composition (H/C ratio) and surface reactivity. Carbonization rates, adjusted using the ratio of extinction measurements at $632$~nm and $1064$~nm, enabled the model to reproduce the time histories of CB volume fraction and $d_p$ in good agreement with TEM measurements.


\end{abstract}

\begin{keyword}
	\KWD Methane Pyrolysis \sep Shock tube \sep Carbon Black \sep Inception \sep Surface Growth \sep Light extinction 
\end{keyword}

\end{frontmatter}

