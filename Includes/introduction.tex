\section{Introduction}

Carbon black (CB) is a critical industrial nanomaterial widely used as a reinforcing agent in rubber and tire~\citep{robertson2021nature} with a global market exceeding \$20 billion~\citep{fortunebusiness2025}. The production of CB remain heavily reliant on furnace processes that suffers from low mass yield and excessive emission, generating  4~kg $\mathrm{CO_2}$ for each kg of product on average~\citep{bansal1993carbon}.

Direct decomposition of methane ($\mathrm{CH_4}$) via thermal plasma is an emerging technology for the coproduction of hydrogen ($\mathrm{H_2}$) and CB, which can achieve 100\% carbon conversion yields without direct $\mathrm{CO_2}$ emission. Achieving specific grades of CB for each target application requires good control over the synthesis process that determines the properties of the manufactured CB such as morphology, composition, specific surface area (SSA) and internal nanostructure~\citep{murphy2001additives}. However, this is challenging due to the complexity of the gas-phase chemistry of and subsequent formation of CB. The processes involved in CB inception and surface growth strongly depend on the local gas temperature~\cite{wang2011formation} and occur on short time scales in the order of few milliseconds~\citep{violi2005relative}. Therefore, it is essential to develop computational models that are benchmarked against reliable measurements in a wide temperature range.

Despite extensive literature on the chemical kinetics of methane pyrolysis (Refs. in \cite{fau2013methane}), the formation of CB from methane pyrolysis is not well understood yet due to the lack of knowledge about pathways leading to CB inception, and the uncertainties in the existing kinetic mechanisms~\citep{agafonov2016unified}. The uncertainty in the chemistry of methane pyrolysis begins with the reaction of methane dissociation~\citep{vlasov2022experimental}, and propagates to small intermediates such as methyl ($\mathrm{CH_3}$)~\citep{wang2016improved}, ethylene ($\mathrm{C_2H_4}$), acetylene ($\mathrm{C_2H_2}$)~\citep{sajid2016shock}. The uncertainty is further amplified for Polycyclic Aromatic Hydrocarbons (PAHs)~\citep{nativel2019shock}, which are widely accepted as main soot precursors~\cite{alfe2009structure}. Prior studies showed that the predicted concentrations of large PAHs can vary by orders of magnitude depending on the reaction mechanism used~\citep{wang2023systematic}. This highlights the need for time-resolved species and particulate measurements under controlled conditions to improve our quantitative understanding of methane conversion to intermediates and carbon flux to CB through inception and surface growth.

Shock tubes are powerful tools for this purpose because a wide range of temperatures (1700-3000~K~\citep{agafonov2016unified}) and pressures (1-100~atm~\citep{shao2020shock}) can be achieved by adjusting the pre-shock pressure ratio; the instant heating of the mixture behind the reflected shock creates an nearly isothermal and isobaric condition over a short test time (1-3 ms) with minimal diffusion and mixing~\citep{eremin2012formation}, which is an ideal setting for kinetic studies of pyrolysis pathways and CB inception and surface growth.

Capturing the rapid decomposition of methane and the subsequent formation of intermediates can be achieved with the high temporal resolution offered by continuous-wave (CW) laser absorption spectroscopy~\citep{pinkowski2019multi}. In this technique, the attenuation of light in an absorbing medium is related to the gas density through the Beer–Lambert law, provided that the absorption cross-section is known. Accurate application, however, is complicated by the strong dependence of the absorption cross-section on temperature, pressure, and gas composition. CB yield and volume fraction can also be quantified with good accuracy from light extinction measurements based on the same principle, as long as the particles remain in the Rayleigh limit and the particle volume fraction is below 10~ppm~\citep{eremin2012formation}. This approach requires prior knowledge of the optical properties of CB, represented by the absorption function $E(m)$. 

While a constant $E(m)$ value of mature CB is commonly used to retrieve volume fractions from extinction data~\citep{de2008scattering, agafonov2016unified, utsav2017simultaneous}, the optical properties are noticeably different for incipient and mature CB. During maturation, CB particles undergo structural and chemical changes that result in the reduction of hydrogen-to-carbon ratio (H/C) due to carbonization~\cite{kholghy2016core} that increases the density and specific heat~\citep{michelsen2021effects}. The maturity level of CB is linked to the spectral dependence of absorption~\citep{bescond2016soot} quantified by the ratio of $E(m)$ at two different wavelengths~\citep{yon2021revealing} that approaches unity for mature CB; therefore, performing extinction measurements at multiple wavelengths can reduce the uncertainties of $E(m)$ and provide a good measure of CB maturity.

%There are only a few studies reporting simultaneous measurements of gas-phase species and particles during the pyrolysis of ethylene, benzene, and acetylene in shock tubes~\citep{agafonov2011soot, utsav2017simultaneous}, but to the authors’ knowledge, no such work has been carried out for methane pyrolysis. In this study, we performed simultaneous measurements of gaseous species and particulate matter during the pyrolysis of 5\% $\mathrm{CH_4}$ in Ar at post-reflected-shock pressures near 5~atm and temperatures between 1800 and 2500K. In parallel, a computational model coupling detailed chemistry, particle dynamics, and an inception model was developed and validated against mass and energy conservation. 
This study follows on our previous work~\citep{clark2025} on simultaneous measurements of gaseous species and particulate matter during the pyrolysis of 5\% $\mathrm{CH_4}$ at post-reflected-shock pressures near 5~atm and temperatures between 1800 and 2500K. A computational model coupling detailed chemistry with CB inception, surface growth and coagulation was utilized to describe methane-to-CB conversion that successfully captured the measured time-history of CB volume fraction ($f_v$), but predicted an increasing temperature trend for primary particle diameter ($d_p$) in contrast to the $d_p$ obtained from Transmission Electron Microscopy (TEM) images. Here, an alternative formulation for surface growth is employed along with the carbonization model accounting the CB maturity during pyrolysis. The modified model not only improves the prediction of $d_p$ in better agreement with TEM images, but also results in H/C that aligns with $E(m)$ ratio from extinction measurements at 633 and 1064~nm. The combined assessment of species and volume fraction ($f_v$) complemented by $d_p$ measurements can be used to constrain the CB inception flux and surface growth rates, thereby improving predictions of soot yield and morphology. The refined model provides a predictive tool for process design and optimization of CB properties under varying operating conditions.

